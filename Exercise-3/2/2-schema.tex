%  DOCUMENT CLASS
\documentclass[11pt]{article}

%PACKAGES
\usepackage[utf8]{inputenc}
%\usepackage[ngerman]{babel}
\usepackage[reqno,fleqn]{amsmath}
\setlength\mathindent{10mm}
\usepackage{amssymb}
\usepackage{amsthm}
\usepackage{color}
\usepackage{delarray}
% \usepackage{fancyhdr}
\usepackage{units}
\usepackage{times, eurosym}
\usepackage{verbatim} %Für Verwendung von multiline Comments mittels \begin{comment}...\end{comment}
\usepackage{wasysym} % Für Smileys
\usepackage{gensymb} % Für \degree
\usepackage{graphicx}
\usepackage{tikz-er2}

% FORMATIERUNG
\usepackage[paper=a4paper,left=25mm,right=25mm,top=25mm,bottom=25mm]{geometry}
\usepackage{array}
\usepackage{fancybox} %zum Einrahmen von Formeln
\setlength{\parindent}{0cm}
\setlength{\parskip}{1mm plus1mm minus1mm}

\allowdisplaybreaks[1]


% PAGESTYLE

%MATH SHORTCUTS
\newcommand{\NN}{\mathbb N}
\newcommand{\ZZ}{\mathbb Z}
\newcommand{\QQ}{\mathbb Q}
\newcommand{\RR}{\mathbb R}
\newcommand{\CC}{\mathbb C}
\newcommand{\KK}{\mathbb K}
\newcommand{\U}{\mathbb O}
\newcommand{\eqx}{\overset{!}{=}}
\newcommand{\Det}{\mathrm{Det}}
\newcommand{\Gl}{\mathrm{Gl}}
\newcommand{\diag}{\mathrm{diag}}
\newcommand{\sign}{\mathrm{sign}}
\newcommand{\rang}{\mathrm{rang}}
\newcommand{\cond}{\mathrm{cond}_{\| \cdot \|}}
\newcommand{\conda}{\mathrm{cond}_{\| \cdot \|_1}}
\newcommand{\condb}{\mathrm{cond}_{\| \cdot \|_2}}
\newcommand{\condi}{\mathrm{cond}_{\| \cdot \|_\infty}}
\newcommand{\eps}{\epsilon}

\setlength{\extrarowheight}{1ex}

\newcommand\tab[1][0.5cm]{\hspace*{#1}}

\begin{document}
	
	\thispagestyle{empty}
	
	\usetikzlibrary{positioning}
	\usetikzlibrary{shadows}
	
	\tikzstyle{every entity} = [top color=white, bottom color=blue!30, 
	draw=blue!50!black!100, drop shadow]
	\tikzstyle{every weak entity} = [drop shadow={shadow xshift=.7ex, 
		shadow yshift=-.7ex}]
	\tikzstyle{every attribute} = [top color=white, bottom color=yellow!20, 
	draw=yellow, node distance=1cm, drop shadow]
	\tikzstyle{every relationship} = [top color=white, bottom color=red!20, 
	draw=red!50!black!100, drop shadow]
	\tikzstyle{every isa} = [top color=white, bottom color=green!20, 
	draw=green!50!black!100, drop shadow]
	
	\centering
	\scalebox{.87}{
		\begin{tikzpicture}[node distance=1.5cm, every edge/.style={link}]
		
		\node[entity] (article) {Article};
		\node[attribute] (articleid) [above=of article] {\key{ID}} edge (article);
		\node[attribute] (text) [above right=of article] {Text} edge (article);
		\node[attribute] (title) [above left=of article] {Title} edge (article);
		
		\node[relationship] (assigned) [below right=of article] {categorized as} edge (article);
		\node[entity] (cat) [below right=of assigned] {Category} edge (assigned);
		\node[attribute] [below=of cat] {\key{Name}} edge (cat);
		
		\node[relationship] (linksto) [below left=of article] {links to} edge (article);
		\coordinate[below=of article] (coordinate) edge (linksto) edge (article);
		
		\node[relationship] (subs) [right=4cm of article] {has subcategories} edge (cat);
		\coordinate[above right=of cat] (subcat) edge (subs) edge (cat);
		
		\coordinate[below left=0.5cm  of article, label={*}];
		\coordinate[below=2cm of article, label={*}];
		\coordinate[below right=1cm of article, label={1}];
		\coordinate[above left=0.5cm of cat, label={*}];  
		\coordinate[above=of cat, label={*}]; 
		\coordinate[above right=of cat, label={*}]; 
		
		\end{tikzpicture}
	}
	\begin{itemize}
		\item There might be multiple articles with the same title, therefore we use an ID as key
		\item Categories should be identified with their name, everything else would lead to confusion
	\end{itemize}
	
	
\end{document}
